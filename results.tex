\label{results_chapter}
\section{Abbreviations and terminology used in this chapter}
In this chapter the following additional abbreviations will be used:
\begin{flushleft}
  \renewcommand{\arraystretch}{1.1}
  \begin{tabularx}{\textwidth}{@{}p{18mm}X@{}}
  U. & Uncalibrated \\
  C. & Calibrated \\
  C. A. E. I. & Calibrated after each training iteration \\
  A. L. N. & Added label noise
  \end{tabularx}
\end{flushleft}

We will use the term experiment in this chapter to refer to the application of a calibration model (see section \ref{experimental_design_and_setup_section}) and/or label noise to a DeepProbLog example application.

\section{Evaluation of model accuracy before and after calibration}
In the data in table \ref{tbl:accuracies_table} we make the following observations:
\begin{itemize}
  \item In a majority of the examples (5 out of 9), the experiment associated with the highest final model accuracy was uncalibrated. When calibration yielded the highest accuracy for an example, it was simple post training calibration in all but one experiment.
  \item In the absence of label noise, on average simple post training calibration reduced model accuracy by 12.14\% while calibrating after each training iteration on average reduced model accuracy by 9.40\%.
  \item In the presence of label noise, on average simple post training calibration reduced model accuracy by 0.98\% while calibrating after each training iteration on average reduced model accuracy by 4.26\%.
  \item Adding label noise on average severely reduces model accuracy (by around 10\%), except when post training calibration is used. In that case, accuracy still deteriorates on average, but only by a mere 0.2\%.
\end{itemize}
These trends can also be seen in figure \ref{fig:accuracies}.
\input{Present_calibration_evaluation_results/accuracies_table}
\begin{figure}[htbp!]
  \centering
  \includegraphics[width=\linewidth]{Present_calibration_evaluation_results/accuracies.png}
  \caption[Accuracies of each experiment]{Accuracies of each experiment}
  \label{fig:accuracies}
\end{figure}
\newpage

\section{Evaluation of model NN ECE before and after last calibration}
In the data in table \ref{tbl:ECE_difference_table} we make the following observations:
\begin{enumerate}
  \item Simple post training calibration on average improved average model NN ECE. Without added label noise, the decrease was on average 0.004216. With added label noise, the decrease was on average 0.046372. In other words, the ECE improvement was an order of magnitude better in the presence of label noise.
  \item Calibration after each training iteration on average improved or worsened average model NN ECE depending on whether label noise was present. Without added label noise, average model NN ECE increased or worsened by 0.050734. With added label noise, average model NN ECE decreased or improved on average by 0.025400.
\end{enumerate}
These trends can also be seen in figure \ref{fig:ECE_difference}.
\input{Present_calibration_evaluation_results/ECE_difference_table}
\begin{figure}[htbp!]
  \centering
  \includegraphics[width=\linewidth]{Present_calibration_evaluation_results/ECE_differences.png}
  \caption[Before \& after final model calibration step model NN ECE $\mu$ and $\sigma$ of each experiment]{Before \& after final model calibration step model NN ECE $\mu$ and $\sigma$ of each experiment}
  \label{fig:ECE_difference}
\end{figure}
\newpage

\section{Evaluation of model loss and average model NN ECE co-evolution}
In the data in table \ref{tbl:loss_ECE_evolution_corr_by_example} and in table \ref{tbl:loss_ECE_evolution_corr_by_experiment} we make the following observations:
\begin{enumerate}
  \item There is significant variance in the Pearson correlation between average model NN ECE and model loss over all examples, ranging from medium positive correlations to medium negative correlations. However, the negative correlations tend to be stronger than the positive correlations.
  \item The Pearson correlation between average model NN ECE and model loss over all experiments tends to be small to medium negative, with only calibration after each training iteration showing a small positive correlation when label noise is present.
  \item The general Pearson correlation coefficient between average model NN ECE and model loss is -0.076, indicating a very small yet negative correlation.
\end{enumerate}

\begin{table}
\centering
\caption{Pearson correlation between model loss (cross-entropy loss) and ECE across iterations by example}
\label{tbl:loss_ECE_evolution_corr_by_example}
\begin{tabular}{lr}
\toprule
{} &  Pearson Correlation \\
Example              &                      \\
\midrule
CLUTRR               &            -0.495106 \\
Coins                &             0.159345 \\
Forth/Add            &            -0.272637 \\
Forth/Sort           &             0.466667 \\
Forth/WAP            &            -0.463345 \\
HWF                  &             0.368752 \\
MNIST addition       &             0.096080 \\
Noisy MNIST addition &             0.080443 \\
Poker                &            -0.366020 \\
\bottomrule
\end{tabular}
\end{table}

\begin{table}
\centering
\caption{Pearson correlation between model loss (cross-entropy loss) and ECE across iterations by experiment}
\label{tbl:loss_ECE_evolution_corr_by_experiment}
\begin{tabular}{lr}
\toprule
{} &  Pearson Correlation \\
Experiment         &                      \\
\midrule
C.                 &            -0.318146 \\
C., A. L. N.       &            -0.157269 \\
C.A.E.I.           &            -0.140203 \\
C.A.E.I., A. L. N. &             0.129485 \\
\bottomrule
\end{tabular}
\end{table}

These trends can also be seen in figure \ref{fig:loss_ECE_evolution}.
\begin{figure}[htbp!]
  \centering
  \includegraphics[width=0.8\linewidth]{Present_calibration_evaluation_results/loss_ECE_evolution.png}
  \caption[Co-evolution of average model NN ECE and model loss over iterations by example and experiment]{Co-evolution of average model NN ECE and model loss over iterations by example and experiment}
  \label{fig:loss_ECE_evolution}
\end{figure}
\newpage

\section{Stand-alone DenseNet Double calibration experiment}
Double calibration of a stand-alone DenseNet trained on the CIFAR100 dataset was performed 100 times and the average change in ECE was found to be a decrease of 0.000536 in ECE, indicating that double calibration can slightly improve neural network calibration.